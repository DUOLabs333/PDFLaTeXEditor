\subsection{Arrow commands}

When drawing simple graphs and other illustrations, the use of arrows is
often essential.  There are two arrow commands in \MP{} for
accommodating this need\Dash |drawarrow| and |drawdblarrow|.  Both of
these commands require a path argument.  For example,

\begin{lstlisting}[style=MP]
drawarrow (0,0)--(72,72);
\end{lstlisting}
draws an arrow beginning at |(0,0)| and ending at |(72,72)| along the
line segment connecting these points.

The path argument of both |drawarrow| and |drawdblarrow| need not be
line segmented paths\Dash they may be any \MP{} path.  The only
difference between |drawarrow| and |drawdblarrow| is that |drawarrow|
places an arrow head at the end of the path and |drawdblarrow| places an
arrow head at the beginning and the end of the path.  As an example, to
draw the curved path in \autoref{fig:draw1} with an arrow head at the
end of the path (i.e., at |z3|), the following command can be used

\begin{lstlisting}[style=MP]
drawarrow z1{right}..z2{dir 45}..{up}z3;
\end{lstlisting}
and is illustrated in \autoref{fig:draw2}.

\begin{figure}
  \begin{withattachment}{draw.mp}
    \centering
    \includegraphics{draw-2.mps}
  \end{withattachment}
  \caption{Using \texttt{drawarrow} along a path}
  \label{fig:draw2}
\end{figure}
