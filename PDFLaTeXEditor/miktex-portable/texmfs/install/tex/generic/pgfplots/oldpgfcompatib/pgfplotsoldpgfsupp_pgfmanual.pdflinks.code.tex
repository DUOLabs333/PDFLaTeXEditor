%%%%%%%%%%%%%%%%%%%%%%%%%%%%%%%%%%%%%%%%%%%%%%%%%
%%% This file is a copy of some part of PGF/Tikz.
%%% It has been copied here to provide :
%%%  - compatibility with older PGF versions
%%%  - availability of PGF contributions by Christian Feuersaenger
%%%    which are necessary or helpful for pgfplots.
%%%
%%% For reasons of simplicity, I have copied the whole file, including own contributions AND
%%% PGF parts. The copyrights are as they appear in PGF.
%%%
%%% Note that pgfplots has compatible licenses.
%%% 
%%% This copy has been modified in the following ways:
%%%  - nested \input commands have been updated
%%%  
%
% Support for the contents of this file will NOT be done by the PGF/TikZ team. 
% Please contact the author and/or maintainer of pgfplots (Christian Feuersaenger) if you need assistance in conjunction
% with the deployment of this patch or partial content of PGF. Note that the author and/or maintainer of pgfplots has no obligation to fix anything:
% This file comes without any warranty as the rest of pgfplots; there is no obligation for help.
%%%%%%%%%%%%%%%%%%%%%%%%%%%%%%%%%%%%%%%%%%%%%%%%%%
%%% Date of this copy: Sa 7. Dez 20:58:23 CET 2013 %%%



%--------------------------------------------
%
% Package pgfplots
%
% Provides a user-friendly interface to create function plots (normal
% plots, semi-logplots and double-logplots).
% 
% It is based on Till Tantau's PGF package.
%
% Copyright 2007/2008/2009 by Christian Feuersänger.
%
% This program is free software: you can redistribute it and/or modify
% it under the terms of the GNU General Public License as published by
% the Free Software Foundation, either version 3 of the License, or
% (at your option) any later version.
% 
% This program is distributed in the hope that it will be useful,
% but WITHOUT ANY WARRANTY; without even the implied warranty of
% MERCHANTABILITY or FITNESS FOR A PARTICULAR PURPOSE.  See the
% GNU General Public License for more details.
% 
% You should have received a copy of the GNU General Public License
% along with this program.  If not, see <http://www.gnu.org/licenses/>.
%
%--------------------------------------------

% This file provides an interface to the
% pgfmanual.prettyprint.code.tex thing -- it allows to 
%  *generate pdf crossrefs inside of codeexamples automatically*
% without any user input.
%
% Thus, you write
% \begin{codeexample}[]
% \begin{tikzpicture}[options]
% \end{tikzpicture}
% \end{codeexample}
% and pdf cross references to the definitions of 'tikzpicture'
% and any options will be generated automatically.
%
% Furthermore, pdf cross references will be generated for everything
% within vertical bars, |....|.
%
%
%
%
%
% The only necessary thing is that \pgfmanualpdflabel has been called
% for every (fully qualified) key, control sequence, environment or
% whatever.

\newif\ifpgfmanualpdfwarnings
\pgfmanualpdfwarningstrue

\newif\ifpgfmanualshowlabels

\pgfkeys{%
	/codeexample/prettyprint/cs arguments/pgfkeys/.initial=1,
	/codeexample/prettyprint/cs/pgfkeys/.code 2 args={\pgfmanualpdfref{#1}{#1}\{\pgfmanualprettyprintpgfkeys{#2}\pgfmanualclosebrace},
	%
	/codeexample/prettyprint/autolinks/.style={%
		/codeexample/prettyprint/key name/.code={\pgfmanualpdfref{##1}{##1}},
		/codeexample/prettyprint/key name with handler/.code 2 args={\pgfmanualpdfref{##1}{##1}/\pgfmanualpdfref{/handlers/##2}{##2}},
		/codeexample/prettyprint/key value display only/.code={\pgfmanualprettyprintcode{##1}},
		/codeexample/prettyprint/cs/.code={\pgfmanualpdfref{##1}{##1}},
		/codeexample/prettyprint/cs with args/.code 2 args={\pgfmanualpdfref{##1}{##1}\{\pgfmanualprettyprintcode{##2}\pgfmanualclosebrace},
		/codeexample/prettyprint/cs arguments/pgfkeys/.initial=1,
		/codeexample/prettyprint/cs/pgfkeys/.code 2 args={\pgfmanualpdfref{##1}{##1}\{\pgfmanualprettyprintpgfkeys{##2}\pgfmanualclosebrace},
		/codeexample/prettyprint/cs arguments/begin/.initial=1,
		/codeexample/prettyprint/cs/begin/.code 2 args={##1\{\pgfmanualpdfref{##2}{##2}\pgfmanualclosebrace},
		/codeexample/prettyprint/cs arguments/end/.initial=1,
		/codeexample/prettyprint/cs/end/.code 2 args={##1\{\pgfmanualpdfref{##2}{##2}\pgfmanualclosebrace},
		/codeexample/prettyprint/word/.code={\begingroup\pgfkeyssetvalue{/pdflinks/search key prefixes in}{}\pgfmanualpdfref{##1}{##1}\endgroup},
		/codeexample/prettyprint/point/.code={##1},%
		/codeexample/prettyprint/point with cs/.code 2 args={(\pgfmanualpdfref{##1}{##1}:##2},%
	},%
	/codeexample/prettyprint/autolinks,
}%

\pgfkeys{
	%
	% Enables or disables the parsing of codeexamples.
	/pdflinks/codeexample links/.is if=pgfmanualprettyenabled,
	/pdflinks/codeexample links/.default=true,
	%
	% whenever an unqualified key is found, the following key prefix
	% list is tried to find a match.
	/pdflinks/search key prefixes in/.initial={/tikz/,/pgf/},
	%
	% Enables or disables warnings for failed auto links:
	/pdflinks/warnings/.is if=pgfmanualpdfwarnings,
	/pdflinks/warnings/.default=true,
	%
	% Shows the autogenerated labels. This is useful to check if the
	% 'search key prefixes in' worked as it ought to.
	/pdflinks/show labels/.is if=pgfmanualshowlabels,
	/pdflinks/show labels/.default=true,
	/pdflinks/show labels=false,
	% will be invoked with '#1' set to the generated label.
	/pdflinks/show labels code/.code={
		\hbox to 0pt{%
			\vbox to 0pt{\hsize=0pt
				\vskip-\baselineskip
				\hbox to \hsize{%
					\hss
					{\footnotesize\ttfamily\textcolor{red}{#1}}%
					\hss
				}%
				\vss
			}%
			\vbox to 0pt{\hsize=0pt
				\vss
				\hbox to \hsize{%
					\hss
					{\footnotesize\ttfamily\textcolor{red}{$\vert$}}%
					\hss
				}%
			}%
			\vsize=0pt
		}%
	},
	%
	% the link prefix written to the pdf file:
	/pdflinks/internal link prefix/.initial=pgf,
}

\begingroup
  \catcode`\_=12
  \gdef\pgfmanualpdf@underscore{_}%
  \catcode`\ =13\relax\gdef\pgfmanualpdf@install@active@space{\def {\space}}%
\endgroup

\gdef\pgfmanualpdf@installreplacements{%
	\def\marg##1{{##1}}%
	\def\oarg##1{[##1]}%
	\def\meta##1{<##1>}%
	\def\x{x}%
	\def\textbackslash{<CS>}%
	\def\\{\textbackslash}%
	\def\space{:}%
	\edef\ {\space}%
	\edef\SPACE{\` \relax}%
	\ifnum\the\catcode`\ =13 %
		\pgfmanualpdf@install@active@space
	\fi
	\edef\#{}%
	\def\printanat{@}%
	\def\protect{}%
	\def\textasciicircum{o}%
        \def\_{\pgfmanualpdf@underscore}%
	\expandafter\edef\pgfmanual@verb@activebar{\pgfmanual@verb@bar}%
}%

% Defines a new pdf cross ref label for use with \pgfmanualpdfref.
%
% Usage:
% 	\pgfmanualpdflabel{<label>}{<text>}
% #1: the label.
% The text #2 will be shown in the resulting pdf (if it is not empty).
%
% There is also support for catcode changes if <label> contains
% something which shouldn't be written as-is into .aux files:
% 	\pgfmanualpdflabel[\catcode`\|=12 ]{|-}{}
% -> this will write
% \begingroup \catcode `\|=12 
%   <code to deal with the label |-  >
% \endgroup 
% into the .aux file.
%
\def\pgfmanualpdflabel{\pgfutil@ifnextchar[{\pgfmanualpdflabel@opt}{\pgfmanualpdflabel@opt[]}}%
\def\pgfmanualpdflabel@opt[#1]#2#3{%
	\begingroup
	%
	\def\pgfmanualpdf@catcode{#1}%
	\pgfmanualpdf@catcode
	%
	\pgfmanualpdflabel@generate{#2}{#3}%
	%
	% this is pgfplots-specific: pgfplots supports generic styles which
	% contain '\x' where '\x' iterates through 'x,y,z'.
	\pgfutil@in@\x{#2}%
	\ifpgfutil@in@
		\def\x{y}%
		\pgfmanualpdflabel@generate{#2}{#3}%
		\def\x{z}%
		\pgfmanualpdflabel@generate{#2}{#3}%
	\fi
	\endgroup
}%
\def\pgfmanualpdflabel@generate#1#2{%
	\pgfmanual@handlespeciallabeltokens@in{#1}%
	%
	\def\pgfmanualpdflabel@generateone{0}%
	\pgfutil@ifundefined{pgfd@lbl@\pgfmanualpdflabel@@}{%
		% ok, no such label is known.
		\def\pgfmanualpdflabel@generateone{1}%
	}{%
		\if\csname pgfd@lbl@\pgfmanualpdflabel@@\endcsname a% "a"ux
			% ah, it is "just" known from a previous run, but there is
			% no code in the pdf! Write it!
			\def\pgfmanualpdflabel@generateone{1}%
		\else
			% ok, we already wrote one before. Skip.
		\fi
	}%
	\if\pgfmanualpdflabel@generateone1%
		\ifpgfmanualshowlabels
			\pgfkeysvalueof{/pdflinks/show labels code/.@cmd}{\pgfmanualpdflabel@@}\pgfeov
		\fi
		%
		\if@filesw
		\ifx\pgfmanualpdf@catcode\pgfutil@empty
		\else
			\toks0=\expandafter{\pgfmanualpdf@catcode}%
			\immediate\write\@auxout{%
				\noexpand\begingroup
				\the\toks0
			}%
		\fi
		\immediate\write\@auxout{%
			\noexpand\expandafter\noexpand\gdef
			\noexpand\csname pgfd@lbl@\pgfmanualpdflabel@@\noexpand\endcsname{a}% a = known in "a"ux file
		}%
		\ifx\pgfmanualpdf@catcode\pgfutil@empty
		\else
			\immediate\write\@auxout{\noexpand\endgroup}%
		\fi
		\fi
		\expandafter\gdef\csname pgfd@lbl@\pgfmanualpdflabel@@\endcsname{w}% 1. remember the label AND remember that we "w"rote it into the pdf.
		\edef\pgfmanualpdflabel@@{\pgfkeysvalueof{/pdflinks/internal link prefix}.\pgfmanualpdflabel@@}%
		\expandafter\hypertarget\expandafter{\pgfmanualpdflabel@@}{#2}%
	\else
		#2%
	\fi
}%

% A pdf reference to label `#1' with (TeX) text `#2'.
% @see also \verbpdfref.
\def\pgfmanualpdfref#1#2{%
	\begingroup
	\pgfmanual@handlespeciallabeltokens@in{#1}%
	%
	\ifcsname pgfd@lbl@\pgfmanualpdflabel@@\endcsname
	\else
		\global\let\pgfmanual@glob=\pgfmanualpdflabel@@
		\def\pgfmanual@tempa{\foreach \prefix in }%
		\pgfkeysgetvalue{/pdflinks/search key prefixes in}\pgfmanual@tempb
		\expandafter\pgfmanual@tempa\expandafter{\pgfmanual@tempb}{%
			\edef\pgfmanualpdflabel@@{\prefix\pgfmanualpdflabel@@}%
			\expandafter\pgfmanual@handlespeciallabeltokens@in\expandafter{\pgfmanualpdflabel@@}%
			\ifcsname pgfd@lbl@\pgfmanualpdflabel@@\endcsname
				\xdef\pgfmanual@glob{\pgfmanualpdflabel@@}%
				\breakforeach
			\fi
		}%
		\let\pgfmanualpdflabel@@=\pgfmanual@glob
		\ifcsname pgfd@lbl@\pgfmanualpdflabel@@\endcsname
		\else
			\ifpgfmanualpdfwarnings
				\begingroup
					\toks0={#1}%
					\pgfmanual@warning{pgfmanualpdfref{\the\toks0 }: target label does not exist.}%
				\endgroup
			\fi
			#2%
			\let\pgfmanualpdflabel@@=\pgfutil@empty
		\fi
	\fi
	\ifx\pgfmanualpdflabel@@\pgfutil@empty
	\else
		\expandafter\pgfmanualpdfref@\expandafter{\pgfmanualpdflabel@@}{#2}%
	\fi
	\endgroup
}%
\def\pgfmanualpdfref@#1#2{%
	\pgfkeysgetvalue{/pdflinks/internal link prefix}\pgfmanual@temp
	\expandafter\hyperlink\expandafter{\pgfmanual@temp.#1}{#2}%
	\ifpgfmanualshowlabels
		\pgfkeysvalueof{/pdflinks/show labels code/.@cmd}{#1}\pgfeov
	\fi
}%

% Handles special tokens in a pdf label which should be treated with
% care.
%
% For example, backslashes might produce problems. 
% This occurs quite frequently with automatically generated hyperrefs 
% inside of codeexamples where \pgfmanualpdfref will be invoked - 
% there, we get the catcode 12 backslashes. 
% Check for them!
%
% #1: a token list which shall be used either as cross ref or as
% label.
%
% On output, the macro \pgfmanualpdflabel@@ will be '\edef'ed to the
% new, possibly modified value.
\def\pgfmanual@handlespeciallabeltokens@in#1{%
	\begingroup
	\pgfmanualpdf@installreplacements
	\expandafter\pgfutil@in@\pgfmanual@pretty@backslash{#1}%
	\ifpgfutil@in@
		% assume the backslash is the first char and substitute it:
		\pgfmanualpdfref@substitute@backslash#1\relax
	\else
		\edef\pgfmanualpdflabel@@{#1}%
	\fi
	\def\pgfmanual@tmp{\pgfutilstrreplace{ }{\space}}%
	\expandafter\pgfmanual@tmp\expandafter{\pgfmanualpdflabel@@}%
	\edef\pgfmanualpdflabel@@{\pgfretval}%
	\pgfmath@smuggleone\pgfmanualpdflabel@@
	\endgroup
}%

\expandafter\def\expandafter\pgfmanualpdfref@substitute@backslash\expandafter#\expandafter1\pgfmanual@pretty@backslash#2\relax{%
	\edef\pgfmanualpdflabel@@{#1\textbackslash #2}%
}%

% Typesets '#1' in red,\texttt like every declaration. It will also
% generate a pdf cross ref anchor for #1.
%
% WARNING: this changes catcodes! In case this is not acceptable in
% your context, you will need to generate a \pgfmanualpdflabel
% manually.
%
% \declareandlabel{\controlsequence}  can be used as |\controlsequence|
\def\declareandlabel{%
	\begingroup
	\pgfmanual@verb@preparecatcodes@
	\def\pgfmanualprettyprinterhandlecollectedargs##1{%
		\pgfmanualpdflabel##1{\texttt{\declare##1}}% mark: '##1' contains already braces.
		\endgroup
	}%
	\pgfmanualprettyprintercollectargcount1{\relax}%
}

% 
% \verbpdfref{\controlsequence more stuff}
% is the same as writing |\controlsequence more stuff|, but the
% *complete* argument is supposed to be one label.
% 
% The difference to \pgfmanualpdfref{...}{} is that the argument is
% supposed to be verbatim text.
\def\verbpdfref{%
	\begingroup
	\pgfmanual@verb@preparecatcodes@
	\def\pgfmanualprettyprinterhandlecollectedargs##1{%
		\pgfmanualpdfref##1{\texttt{##1}}% mark: '##1' contains already braces.
		\endgroup
	}%
	\pgfmanualprettyprintercollectargcount1{\relax}%
}

% Prepare active vertical bars, |....| for auto-pretty cross
% referencing.
%
% Example: 
%  |\pgfkeys| -> will generate a hyperref!
{
	\catcode`\|=12
	\gdef\pgfmanual@verb@bar{|}%
%	\gdef\pgfmanual@verb@collect#1|{%
%		% this command will also handle control sequences.
%		\texttt{\pgfmanualprettyprintpgfkeys{#1}}%
%		\endgroup
%	}%
	\catcode`\|=13
	\gdef\pgfmanual@verb@activebar{|}%
}
\def\pgfmanual@verb{%
	\begingroup
	\pgfmanual@verb@preparecatcodes@
	\toksdef\t@pgfmanual@verb=0
	\t@pgfmanual@verb={}%
	\pgfmanual@verb@collect
}
% this version of \pgfmanual@verb@collect is less efficient than the
% one uncommented above. BUT: it can auto-detect the case when
% |...| has been provided somewhere where I can't change catcodes!
% The other one would simply fail to compile.
\def\pgfmanual@verb@collect#1{%
	\def\pgfmanual@temp{#1}%
	\ifx\pgfmanual@temp\pgfmanual@verb@bar
		% ok, finish:
		\edef\pgfmanual@verb@collect@next{%
			% this command will also handle control sequences.
			\noexpand\endgroup
			\noexpand\texttt{\noexpand\pgfmanualprettyprintpgfkeys{\the\t@pgfmanual@verb}}%
		}%
	\else
		\ifx\pgfmanual@temp\pgfmanual@verb@activebar
			% ohoh... that should not happen! It means someone invoked
			% |...| within an argument; I couln't change catcodes.
			% Ok, resort to a simple fallback solution.
			% FIXME : I have just realized that THIS DOESN'T PRESERVE SPACES
			\edef\pgfmanual@verb@collect@next{%
				\noexpand\endgroup
				\noexpand\texttt{\the\t@pgfmanual@verb}%
			}%
		\else
			\t@pgfmanual@verb=\expandafter{\the\t@pgfmanual@verb #1}%
			\let\pgfmanual@verb@collect@next=\pgfmanual@verb@collect
		\fi
	\fi
	\pgfmanual@verb@collect@next
}%

\AtBeginDocument{%
	\ifpgfmanualprettyenabled
		\catcode`\|=13
		\expandafter\let\pgfmanual@verb@activebar=\pgfmanual@verb
	\fi
}%

\def\pgfmanual@verb@preparecatcodes@{%
	\let\do\@makeother%
	\dospecials%
	\catcode`\%=12 % THATS IMPORTANT! Do *not* handle comments!
	% these catcodes are expected by the pretty printer...
	%\catcode`\^^M=13
	\catcode`\ =13
	\catcode`\^^I=13
	\expandafter\def\pgfmanual@pretty@activespace{\space}%
	\expandafter\def\pgfmanual@pretty@activetab{\space\space\space\space}%
}%
\endinput
% vi: ts=4 sw=4
